%package list
\documentclass{article}
\usepackage[top=3cm, bottom=3cm, outer=3cm, inner=3cm]{geometry}
\usepackage{multicol}
\usepackage{listings}
\usepackage[utf8]{inputenc}
\usepackage{graphicx}
\usepackage{url}
%\usepackage{cite}
\usepackage{hyperref}
\usepackage{array}
%\usepackage{multicol}
\newcolumntype{x}[1]{>{\centering\arraybackslash\hspace{0pt}}p{#1}}
\usepackage{natbib} 
\usepackage{pdfpages}
\usepackage{multirow}   
\usepackage[normalem]{ulem}
\useunder{\uline}{\ul}{}
\usepackage{svg}
\usepackage{xcolor}
\usepackage{listings}
\lstdefinestyle{ascii-tree}{
    literate={├}{|}1 {─}{--}1 {└}{+}1 
  }
\lstset{basicstyle=\ttfamily,
  showstringspaces=false,
  commentstyle=\color{red},
  keywordstyle=\color{blue}
}
%\usepackage{booktabs}
\usepackage{caption}
\usepackage{subcaption}
\usepackage{float}
\usepackage{array}

\newcolumntype{M}[1]{>{\centering\arraybackslash}m{#1}}
\newcolumntype{N}{@{}m{0pt}@{}}


%%%%%%%%%%%%%%%%%%%%%%%%%%%%%%%%%%%%%%%%%%%%%%%%%%%%%%%%%%%%%%%%%%%%%%%%%%%%
%%%%%%%%%%%%%%%%%%%%%%%%%%%%%%%%%%%%%%%%%%%%%%%%%%%%%%%%%%%%%%%%%%%%%%%%%%%%
\newcommand{\itemEmail}{ shanccom@unsa.edu.pe
}
\newcommand{\itemStudent}{ Sergio Hancco Mullisaca }
\newcommand{\itemCourse}{Programacion Web 2}
\newcommand{\itemCourseCode}{}
\newcommand{\itemSemester}{II}
\newcommand{\itemUniversity}{Universidad Nacional de San Agustín de Arequipa}
\newcommand{\itemFaculty}{Facultad de Ingeniería de Producción y Servicios}
\newcommand{\itemDepartment}{Departamento Académico de Ingeniería de Sistemas e Informática}
\newcommand{\itemSchool}{Escuela Profesional de Ingeniería de Sistemas}
\newcommand{\itemAcademic}{2024 - A}
\newcommand{\itemInput}{Del 29 Mayo 2024}
\newcommand{\itemOutput}{Al 05 Junio 2024}
\newcommand{\itemPracticeNumber}{7}
\newcommand{\itemTheme}{DJANGO II}
%%%%%%%%%%%%%%%%%%%%%%%%%%%%%%%%%%%%%%%%%%%%%%%%%%%%%%%%%%%%%%%%%%%%%%%%%%%%
%%%%%%%%%%%%%%%%%%%%%%%%%%%%%%%%%%%%%%%%%%%%%%%%%%%%%%%%%%%%%%%%%%%%%%%%%%%%

\usepackage[english,spanish]{babel}
\usepackage[utf8]{inputenc}
\AtBeginDocument{\selectlanguage{spanish}}
\renewcommand{\figurename}{Figura}
\renewcommand{\refname}{Referencias}
\renewcommand{\tablename}{Tabla} %esto no funciona cuando se usa babel
\AtBeginDocument{%
	\renewcommand\tablename{Tabla}
}

\usepackage{fancyhdr}
\pagestyle{fancy}
\fancyhf{}
\setlength{\headheight}{30pt}
\renewcommand{\headrulewidth}{1pt}
\renewcommand{\footrulewidth}{1pt}
\fancyhead[L]{\raisebox{-0.2\height}{\includegraphics[width=3cm]{logo_episunsa.png}}}
\begin{figure}
    \centering
    \label{fig:enter-label}
\end{figure}
\fancyhead[C]{\fontsize{7}{7}\selectfont	\itemUniversity \\ \itemFaculty \\ \itemDepartment \\ \itemSchool \\ \textbf{\itemCourse}}
\fancyhead[R]{\raisebox{-0.2\height}{\includegraphics[width=1.2cm]{}}}
\fancyfoot[C]{\itemCourse}
\fancyfoot[R]{Página \thepage}

% para el codigo fuente
\usepackage{listings}
\usepackage{color, colortbl}
\definecolor{dkgreen}{rgb}{0,0.6,0}
\definecolor{gray}{rgb}{0.5,0.5,0.5}
\definecolor{mauve}{rgb}{0.58,0,0.82}
\definecolor{codebackground}{rgb}{0.95, 0.95, 0.92}
\definecolor{tablebackground}{rgb}{0.8, 0, 0}

\lstset{frame=tb,
	language=bash,
	aboveskip=3mm,
	belowskip=3mm,
	showstringspaces=false,
	columns=flexible,
	basicstyle={\small\ttfamily},
	numbers=none,
	numberstyle=\tiny\color{gray},
	keywordstyle=\color{blue},
	commentstyle=\color{dkgreen},
	stringstyle=\color{mauve},
	breaklines=true,
	breakatwhitespace=true,
	tabsize=3,
	backgroundcolor= \color{codebackground},
}

\begin{document}
	
	\vspace*{10px}
	
	\begin{center}	
		\fontsize{17}{17} \textbf{ Informe de Laboratorio \itemPracticeNumber}
	\end{center}
	\centerline{\textbf{\Large Tema: \itemTheme}}
	%\vspace*{0.5cm}	

	\begin{flushright}
		\begin{tabular}{|M{2.5cm}|N|}
			\hline 
			\rowcolor{tablebackground}
			\color{white} \textbf{Nota}  \\
			\hline 
			     \\[30pt]
			\hline 			
		\end{tabular}
	\end{flushright}	

	\begin{table}[H]
		\begin{tabular}{|x{4.7cm}|x{4.8cm}|x{4.8cm}|}
			\hline 
			\rowcolor{tablebackground}
			\color{white} \textbf{Estudiante} & \color{white}\textbf{Escuela}  & \color{white}\textbf{Asignatura}   \\
			\hline 
			{\itemStudent \par \itemEmail} & \itemSchool & {\itemCourse \par Semestre: \itemSemester \par Código: \itemCourseCode}     \\
			\hline 			
		\end{tabular}
	\end{table}		
	
	\begin{table}[H]
		\begin{tabular}{|x{4.7cm}|x{4.8cm}|x{4.8cm}|}
			\hline 
			\rowcolor{tablebackground}
			\color{white}\textbf{Laboratorio} & \color{white}\textbf{Tema}  & \color{white}\textbf{Duración}   \\
			\hline 
			\itemPracticeNumber  & \itemTheme & 04 horas   \\
			\hline 
		\end{tabular}
	\end{table}
	
	\begin{table}[H]
		\begin{tabular}{|x{4.7cm}|x{4.8cm}|x{4.8cm}|}
			\hline 
			\rowcolor{tablebackground}
			\color{white}\textbf{Semestre académico} & \color{white}\textbf{Fecha de inicio}  & \color{white}\textbf{Fecha de entrega}   \\
			\hline 
			\itemAcademic & \itemInput &  \itemOutput  \\
			\hline 
		\end{tabular}
	\end{table}
	
	\section{Tarea}
	\begin{itemize}		
		\item Informe de laboratorio
            \item Video en Flip
		\item Ejercicios Propuestos
        
	\end{itemize}
		
	\section{Equipos, materiales y temas utilizados}
	\begin{itemize}
		\item VS
		\item Git 2.39.2.
		\item Cuenta en GitHub con el correo institucional.
	\end{itemize}
    \clearpage
    
	\section{URL de Repositorio Github}
	\begin{itemize}
        \item URL del video en yt.
		\item \url{https://youtu.be/jTnHXFnj-ZI}
        \item URL del video en flip.
		\item \url{https://flip.com/s/sdBCVHN1UHXB}
        \item URL del GITHUB.
            \item \url{https://github.com/shanccom/Programacion_Web_2.git}
	\end{itemize}
	
	\section{Actividades}
	\subsection{EJERCICIO PROPUESTO}
        \item CREACION ENTORNO VIRTUAL CON GITIGNORE
        \newline \newline
        \includegraphics[width=0.8\textwidth,keepaspectratio]{Creacion entorno virtual.png}

        \item ARCHIVOS MODIFICADOS (TELUSKO)
        
        \begin{lstlisting}[language=Python, caption=settings.py]
INSTALLED_APPS = [
    'travello.apps.TravelloConfig',
    'django.contrib.admin',
    'django.contrib.auth',
    'django.contrib.contenttypes',
    'django.contrib.sessions',
    'django.contrib.messages',
    'django.contrib.staticfiles',
]

MIDDLEWARE = [
    'django.middleware.security.SecurityMiddleware',
    'django.contrib.sessions.middleware.SessionMiddleware',
    'django.middleware.common.CommonMiddleware',
    'django.middleware.csrf.CsrfViewMiddleware',
    'django.contrib.auth.middleware.AuthenticationMiddleware',
    'django.contrib.messages.middleware.MessageMiddleware',
    'django.middleware.clickjacking.XFrameOptionsMiddleware',
]

ROOT_URLCONF = 'telusko.urls'

TEMPLATES = [
    {
        'BACKEND': 'django.template.backends.django.DjangoTemplates',
        'DIRS': [os.path.join(BASE_DIR, 'templates')],
        'APP_DIRS': True,
        'OPTIONS': {
            'context_processors': [
                'django.template.context_processors.debug',
                'django.template.context_processors.request',
                'django.contrib.auth.context_processors.auth',
                'django.contrib.messages.context_processors.messages',
            ],
        },
    },
]

WSGI_APPLICATION = 'telusko.wsgi.application'


# Database
# https://docs.djangoproject.com/en/2.2/ref/settings/#databases

DATABASES = {
    'default': {
        'ENGINE': 'django.db.backends.postgresql',
        'NAME': 'telusko',
        'USER': 'postgres',
        'PASSWORD': '123',
        'HOST': 'localhost'
    }
}

        \end{lstlisting}
        
        \begin{lstlisting}[language=Python, caption=urls.py]
from django.contrib import admin
from django.urls import path, include
from django.conf import settings
from django.conf.urls.static import static


urlpatterns = [
    path('', include('travello.urls')),
    path('admin/', admin.site.urls),
    path('accounts/',include('accounts.urls'))
] 

urlpatterns = urlpatterns + static(settings.MEDIA_URL, document_root=settings.MEDIA_ROOT)
        \end{lstlisting}  

        \begin{lstlisting}[language=Python, caption=wsgi.py]
import os

from django.core.wsgi import get_wsgi_application

os.environ.setdefault('DJANGO_SETTINGS_MODULE', 'telusko.settings')

application = get_wsgi_application()

        \end{lstlisting}  
        \item ARCHIVOS MODIFICADOS (CALC)
        \begin{lstlisting}[language=Python, caption=apps.py]
from django.apps import AppConfig


class CalcConfig(AppConfig):
    default_auto_field = 'django.db.models.BigAutoField'
    name = 'calc'

        \end{lstlisting}  
        \begin{lstlisting}[language=Python, caption=urls.py]
from django.urls import path

from . import views

urlpatterns = [path("", views.home, name="home"), path("add", views.add, name="add")]
        \end{lstlisting}  
        \begin{lstlisting}[language=Python, caption=views.py]
from django.shortcuts import render
from django.http import HttpResponse

# Create your views here.


def home(request):
    return render(request, "home.html", {"name": "Navin"})


def add(request):

    val1 = int(request.POST["num1"])
    val2 = int(request.POST["num2"])
    res = val1 + val2

    return render(request, "result.html", {"result": res})

        \end{lstlisting}  
        
        \item ARCHIVOS MODIFICADOS (TRAVELLO-APP)
        
        \begin{lstlisting}[language=Python, caption=admin.py]
from django.contrib import admin
from .models import Destination

admin.site.register(Destination)
        \end{lstlisting} 

        \begin{lstlisting}[language=Python, caption=apps.py]
from django.apps import AppConfig

class TravelloConfig(AppConfig):
    name = 'travello'

        \end{lstlisting} 
        
        \begin{lstlisting}[language=Python, caption=models.py]
from django.db import models

class Destination(models.Model):
    
    name = models.CharField(max_length=100)
    img = models.ImageField(upload_to='pics')
    desc = models.TextField()
    price = models.IntegerField()
    offer = models.BooleanField(default=False)
        \end{lstlisting} 

        \begin{lstlisting}[language=Python, caption=urls.py]
from django.urls import path

from . import views

urlpatterns = [path("", views.index, name="index")]

        \end{lstlisting} 

\begin{lstlisting}[language=Python, caption=views.py]
from django.shortcuts import render
from .models import Destination
# Create your views here.


def index(request):

    dests = Destination.objects.all()

    return render(request, "index.html", {'dests': dests})
    
        \end{lstlisting} 

        \newline\newline\newline\newline\newline\newline\newline\newline
        \item IMAGENES-PRUEBAS

        \item INSTALACION DE POSTRGRESQL - PGADMIN
        \newline\newline\newline
        \includegraphics[width=1\textwidth,keepaspectratio]{IMAGENES/Instalacion de postgree y pgadmin.png}

        \item MIGRACIONES
        \newline\newline\newline
        \includegraphics[width=1\textwidth,keepaspectratio]{IMAGENES/migracion de tablas.png}

        \item TABLAS EN PGADMIN
        \newline\newline\newline
        \includegraphics[width=0.7\textwidth,keepaspectratio]{IMAGENES/tablas.png}

        \item PRUEBAS DE FUNCIONAMIENTO (PRUEBA DE SUMA)
        \newline\newline\newline
        \includegraphics[width=1\textwidth,keepaspectratio]{IMAGENES/Prueba(suma de numeros).png}

        \item PRUEBAS DE FUNCIONAMIENTO (PRUEBA DE LA PAGINA EN SI)
        \newline\newline\newline
        \includegraphics[width=1\textwidth,keepaspectratio]{IMAGENES/prueba(html-css).png}
















































        

    \clearpage

	\section{\textcolor{red}{Rúbricas}}
	
	\subsection{\textcolor{red}{Entregable Informe}}
	\begin{table}[H]
		\caption{Tipo de Informe}
		\setlength{\tabcolsep}{0.5em} % for the horizontal padding
		{\renewcommand{\arraystretch}{1.5}% for the vertical padding
		\begin{tabular}{|p{3cm}|p{12cm}|}
			\hline
			\multicolumn{2}{|c|}{\textbf{\textcolor{red}{Informe}}}  \\
			\hline 
			\textbf{\textcolor{red}{Latex}} & \textcolor{blue}{El informe está en formato PDF desde Latex,  con un formato limpio (buena presentación) y facil de leer.}   \\ 
			\hline 
			
			
		\end{tabular}
	}
	\end{table}
	

	
	\subsection{\textcolor{red}{Rúbrica para el contenido del Informe y demostración}}
	\begin{itemize}			
		\item El alumno debe marcar o dejar en blanco en celdas de la columna \textbf{Checklist} si cumplio con el ítem correspondiente.
		\item Si un alumno supera la fecha de entrega,  su calificación será sobre la nota mínima aprobada, siempre y cuando cumpla con todos lo items.
		\item El alumno debe autocalificarse en la columna \textbf{Estudiante} de acuerdo a la siguiente tabla:
	
		\begin{table}[ht]
			\caption{Niveles de desempeño}
			\begin{center}
			\begin{tabular}{ccccc}
    			\hline
    			 & \multicolumn{4}{c}{Nivel}\\
    			\cline{1-5}
    			\textbf{Puntos} & Insatisfactorio 25\%& En Proceso 50\% & Satisfactorio 75\% & Sobresaliente 100\%\\
    			\textbf{2.0}&0.5&1.0&1.5&2.0\\
    			\textbf{4.0}&1.0&2.0&3.0&4.0\\
    		\hline
			\end{tabular}
		\end{center}
	\end{table}	
	
	\end{itemize}
	
	\begin{table}[H]
		\caption{Rúbrica para contenido del Informe y demostración}
		\setlength{\tabcolsep}{0.5em} % for the horizontal padding
		{\renewcommand{\arraystretch}{1.5}% for the vertical padding
		%\begin{center}
		\begin{tabular}{|p{2.7cm}|p{7cm}|x{1.3cm}|p{1.2cm}|p{1.5cm}|p{1.1cm}|}
			\hline
    		\multicolumn{2}{|c|}{Contenido y demostración} & Puntos & Checklist & Estudiante & Profesor\\
			\hline
			\textbf{1. GitHub} & Hay enlace URL activo del directorio para el  laboratorio hacia su repositorio GitHub con código fuente terminado y fácil de revisar. &2 &X &2 & \\ 
			\hline
			\textbf{2. Commits} &  Hay capturas de pantalla de los commits más importantes con sus explicaciones detalladas. (El profesor puede preguntar para refrendar calificación). &4 &X &4 &  \\ 
			\hline 
			\textbf{3. Código fuente} &  Hay porciones de código fuente importantes con numeración y explicaciones detalladas de sus funciones. &2 &X &2 & \\ 
			\hline 
			\textbf{4. Ejecución} & Se incluyen ejecuciones/pruebas del código fuente  explicadas gradualmente. &2 &X &1 & \\ 
			\hline			
			\textbf{5. Pregunta} & Se responde con completitud a la pregunta formulada en la tarea.  (El profesor puede preguntar para refrendar calificación).  &2 &X &2 & \\ 
			\hline	
			\textbf{6. Fechas} & Las fechas de modificación del código fuente estan dentro de los plazos de fecha de entrega establecidos. &2 &X &2 & \\ 
			\hline 
			\textbf{7. Ortografía} & El documento no muestra errores ortográficos. &2 &X &2 & \\ 
			\hline 
			\textbf{8. Madurez} & El Informe muestra de manera general una evolución de la madurez del código fuente,  explicaciones puntuales pero precisas y un acabado impecable.   (El profesor puede preguntar para refrendar calificación).  &4 &X &4 & \\ 
			\hline
			\multicolumn{2}{|c|}{\textbf{Total}} &20 & &19 & \\ 
			\hline
		\end{tabular}
		%\end{center}
		%\label{tab:multicol}
		}
	\end{table}
	
\clearpage
	   
	
%\clearpage
%\bibliographystyle{apalike}
%\bibliographystyle{IEEEtranN}
%\bibliography{bibliography}
			
\end{document}